\usepackage{tikz}  
\usepackage{tikz-3dplot} 
\usepackage{graphicx}
\usepackage{media9}
\usetikzlibrary{positioning}

\usepackage{amsmath}
\usepackage{booktabs}

% change the camera position
\tdplotsetmaincoords{45}{135}

\usepackage{tabularx}

\usepackage{subcaption}

\usepackage{animate}

\usepackage{pgfmath}
\newcommand\randmin{}
\newcommand\randmax{}
\newcommand\randmultof{}
\newcommand\setrand[4]%
{\def\randmin{#1}%
	\def\randmax{#2}%
	\def\randmultof{#3}%
	\pgfmathsetseed{#4}%
}
\newcommand\nextrand
{\pgfmathparse{int(int((rnd*(\randmax-\randmin+1)+\randmin)/\randmultof)*\randmultof)}%
	\xdef\thisrand{\pgfmathresult}%
}

%\usepackage[backend=biber]{biblatex}
%\addbibresource{bib/references.bib}

\newcommand{\wave}{
	\begin{tikzpicture}[xscale=.05,yscale=.2]
	\draw[-,fill=white] plot[domain=0:10*pi,smooth] (\x,{sin(\x r)});
	\end{tikzpicture}
}


\newcommand{\domain}[4]{
	%% spatial,spectral,temporal
	\draw[fill=#4, opacity=.2] (#1,0,0) -- (0,#2,0) -- (0,0,#3) -- (#1,0,0);
}


\newcommand{\radarwave}{
	\begin{tikzpicture}[xscale=.1,yscale=.2]
	\draw[-,fill=white] plot[domain=0:5*pi,smooth] (\x,{sin(\x r)});
	\end{tikzpicture}
}


\newcommand{\earth}{
	\begin{tikzpicture}[baseline=-.25em, inner sep=0]
	\node{\includegraphics[width=8mm]{images/icons/earth}};
	\end{tikzpicture}
}

\newcommand{\sat}{
	\begin{tikzpicture}[baseline=-.25em, inner sep=0]
	\node[rotate=270,anchor=center]{\includegraphics[width=8mm]{images/icons/sat2}};
	\end{tikzpicture}
}


\usepackage[capitalize]{cleveref}
\usepackage[square,sort,comma,numbers]{natbib}

%% this hack seems to be nececessary due to incompatibilities of cvpr template and tikz... -> https://tex.stackexchange.com/questions/398223/tikz-gives-error-command-everyshipouthook-already-defined
%\makeatletter
%\@namedef{ver@everyshi.sty}{}
%\makeatother
%% hackend

\usepackage{tikz}
\usepackage{pgfplots}
\usetikzlibrary{positioning, calc,arrows,arrows.meta, fit}
%\usetikzlibrary{arrows.meta,calc,decorations.markings,math,arrows.meta}
\usepgfplotslibrary{groupplots}
\usepgfplotslibrary{fillbetween}
\usepgfplotslibrary{statistics} % provides boxplots
\usepackage{xfrac}

\newcommand{\tp}{tp}
\newcommand{\tn}{tn}
\newcommand{\fp}{fp}
\newcommand{\fn}{fn}


\usepackage{tumcolors}
\usepackage{tummath}
\newcommand{\yhat}{\hat{\V{y}}}
\newcommand{\ycorrect}{\hat{y}^+}
\newcommand{\thetadelta}{\V{\Theta}_\delta}
\newcommand{\biasdelta}{b_\delta}
\newcommand{\biasclass}{\V{b}_\text{c}}
\newcommand{\thetaclass}{\V{\Theta}_\text{c}}
\newcommand{\thetafeat}{\V{\Theta}_\text{feat}}
\newcommand{\fclass}{f_\text{c}}
\newcommand{\fdelta}{f_\delta}
\newcommand{\ffeat}{f_\text{feat}}
\newcommand{\f}{f}

\newcommand{\rvtime}{T_c} 
\newcommand{\xuptot}{\M{X}_{\rightarrow t}} 
\newcommand{\deltauptot}{\delta_{\rightarrow t}} 
\newcommand{\tstop}{\ensuremath{t_\text{stop}}}
\newcommand{\meantstop}{\ensuremath{\bar{t}_\text{stop}}}
\usepackage[super]{nth}
\usepackage{mathtools}

\definecolor{evalcolor}{HTML}{3F3F3F}
\definecolor{traincolor}{HTML}{B98951}
\definecolor{validcolor}{HTML}{3F4BBE}

\definecolor{fdlcolor}{HTML}{142737}

\usepackage{multimedia}

\colorlet{colortrain}{tumblue}
\colorlet{colorinfer}{tumblack}

\colorlet{earlinesscolor}{tumblue}
\colorlet{accuracycolor}{tumorange}

\colorlet{stdcolor}{tumbluelight}
\colorlet{mediancolor}{tumorange}
\colorlet{meancolor}{tumblue}

\colorlet{b1color}{tumdiagramaubergine}
\colorlet{b2color}{tumdiagramnavyblue}
\colorlet{b3color}{tumdiagramturquoise}
\colorlet{b4color}{tumdiagramgreen}
\colorlet{b5color}{tumdiagramlimegreen}
\colorlet{b6color}{tumdiagramyellow}
\colorlet{b7color}{tumdiagramsand}
\colorlet{b8color}{tumdiagramredorange}
\colorlet{b8Acolor}{tumdiagramred}
\colorlet{b9color}{tumblack}
\colorlet{b10color}{tumblue}
\colorlet{b11color}{tumdiagramdarkred}
\colorlet{b12color}{tumorange}

% atmospheric bands
\colorlet{b1color}{tumblack}%tumdiagramaubergine
\colorlet{b9color}{tumblack}%tumblack
\colorlet{b10color}{tumblack}%tumblue

%visisble bands
\colorlet{b2color}{tumblue}%tumdiagramnavyblue
\colorlet{b3color}{tumblue}%tumdiagramturquoise
\colorlet{b4color}{tumblue}%tumdiagramgreen

% near infrared bands
\colorlet{b5color}{tumdiagramred}%tumdiagramlimegreen
\colorlet{b6color}{tumdiagramred}%tumdiagramyellow
\colorlet{b7color}{tumdiagramred}%tumdiagramsand
\colorlet{b8color}{tumdiagramred}%tumdiagramredorange
\colorlet{b8Acolor}{tumdiagramred}%tumdiagramred

% SWIR bands
\colorlet{b11color}{tumorange}%tumdiagramdarkred
\colorlet{b12color}{tumorange}%tumorange

\colorlet{epsilon0color}{tumorange}
\colorlet{epsilon1color}{tumblue}
\colorlet{epsilon10color}{tumblack}

\colorlet{gridcolor}{tumblue}
\colorlet{activationcolor}{tumorange}

\colorlet{meadowcolor}{tumbluemedium}
\colorlet{wbarleycolor}{tumbluedark}
\colorlet{corncolor}{tumorange}
\colorlet{wheatcolor}{tumgreen}
\colorlet{sbarleycolor}{tumdiagramred}
\colorlet{clovercolor}{tumdiagramturquoise}
\colorlet{triticalecolor}{tumdiagramsand}

\tikzstyle{rnn}=[draw,circle, inner sep=.1em]
\tikzstyle{norm}=[rounded corners,draw]
\tikzstyle{annot}=[rounded corners, fill=tumblue!20]
\tikzstyle{infer}=[-stealth, shorten >=.0em, shorten <=.0em, colorinfer]
\tikzstyle{loss}=[fill=tumblue!10, rounded corners, font=\small]
\tikzstyle{grad}=[colortrain]

\newcommand{\ptoffset}{\varepsilon}

\tikzstyle{test} = [thick]
\tikzstyle{train} = [thin, dotted]

\usepackage[inline]{enumitem}
\setenumerate{label=(\roman*),itemsep=3pt,topsep=3pt}

\setlength{\belowcaptionskip}{-10pt}


\colorlet{traincolor}{tumbluelight}
\colorlet{validcolor}{tumbluedark}
\colorlet{evalcolor}{tumorange}

\colorlet{forwardcolor}{tumblue}
\colorlet{backwardcolor}{tumorange}

% defaultvalue -> might be replaced later
\colorlet{tensorcolor}{forwardcolor}

\colorlet{classcolor}{tumivory}
\colorlet{encodercolor}{tumblue}
\colorlet{encodercolor}{tumblue}
\colorlet{colorblue}{tumblue}
\colorlet{colororange}{tumorange}

\colorlet{colorclassone}{tumblue}
\colorlet{colorclasstwo}{tumblack}
\colorlet{colorclassthree}{tumorange}
\colorlet{colorclassfour}{tumgray}

\colorlet{frh01color}{tumgray}
\colorlet{frh02color}{tumorange}
\colorlet{frh03color}{tumblue}
\colorlet{frh04color}{tumblack}



%\usepackage{media9}

% notation
\newcommand{\MWeight}{\ensuremath{\M{W}}}
\newcommand{\VBias}{\ensuremath{\V{b}}}
\newcommand{\VInput}{\DataVec}
\newcommand{\VHidden}{\ensuremath{\V{h}}}
\newcommand{\FActivation}{\ensuremath{\sigma}}
\newcommand{\VCellState}{\ensuremath{\V{c}}}
\newcommand{\VForgetGate}{\ensuremath{\V{f}}}
\newcommand{\VModulationGate}{\ensuremath{\V{j}}}
\newcommand{\VInputGate}{\ensuremath{\V{i}}}
\newcommand{\VOutputGate}{\ensuremath{\V{o}}}
\newcommand{\concat}[2]{\left[#1 \parallel #2\right]}


\newcommand{\VResetGate}{\ensuremath{\V{r}}}
\newcommand{\VUpdateGate}{\ensuremath{\V{u}}}


%\usepackage{titlesec}
%\titlespacing{\section}{0pt}{10pt}{3pt}


\usetikzlibrary{3d}
\tikzstyle{perspective3d}=[
x={(0.5cm,0.5cm)}, y={(1cm,0cm)}, z={(0cm,1cm)}]


\usetikzlibrary{spy}

\usetikzlibrary{external,pgfplots.dateplot}

\usepackage[eulergreek]{sansmath}
\pgfplotsset{
	y tick label style={/pgf/number format/.cd,%
		scaled y ticks = false,
		set thousands separator={},
		fixed},
	x tick label style={/pgf/number format/.cd,%
		scaled x ticks = false,
		set decimal separator={,},
		fixed},
	tick label style = {font=\scriptsize\sansmath\sffamily},
	every axis label = {
		font=\scriptsize\sansmath\sffamily},
	every axis/.append style={
		axis lines=left, 
		enlargelimits, 
		thick},
	legend style = {font=\scriptsize\sansmath\sffamily, draw=none, rounded corners=0, fill opacity=.5, text opacity=1},
	label style = {font=\scriptsize\sansmath\sffamily},
	grid style={line width=.1pt, draw=gray!10},
	major grid style={line width=.2pt,draw=tumgraylight},
}

%\let\tempone\itemize
%\let\temptwo\enditemize
%\renewenvironment{itemize}{\tempone\addtolength{\itemsep}{-.5\baselineskip}}{\temptwo}

\tikzstyle{circ} = [circle, draw=white, fill=tumblue, inner sep=1pt]
\newcommand{\fcn}{
	\begin{tikzpicture}[scale=0.2, rotate=0, baseline=-.25em, inner sep=1pt]
	\node[circ](a0) at (0,-1){};
	\node[circ](a1) at (0,0){};
	\node[circ](a2) at (0,1){};
	
	\node[circ](b0) at (1,-0.5){};
	\node[circ](b1) at (1,0.5){};
	
	\draw[-] (a0) -- (b0);
	\draw[-] (a1) -- (b0);
	\draw[-] (a2) -- (b0);
	
	\draw[-] (a0) -- (b1);
	\draw[-] (a1) -- (b1);
	\draw[-] (a2) -- (b1);
	
	\end{tikzpicture}
}

\newcommand{\lfcn}[1]{
	\begin{tikzpicture}[scale=#1, rotate=0, baseline=-.25em, inner sep=1pt]
	\node[circle, draw=white, fill=tumblue, inner sep=3](a0) at (0,-1){};
	\node[circle, draw=white, fill=tumblue,  inner sep=3](a1) at (0,0){};
	\node[circle, draw=white, fill=tumblue,  inner sep=3](a2) at (0,1){};
	
	\node[circle, draw=white, fill=tumblue,  inner sep=3](b0) at (1,-0.5){};
	\node[circle, draw=white, fill=tumblue,  inner sep=3](b1) at (1,0.5){};
	
	\draw[-] (a0) -- (b0);
	\draw[-] (a1) -- (b0);
	\draw[-] (a2) -- (b0);
	
	\draw[-] (a0) -- (b1);
	\draw[-] (a1) -- (b1);
	\draw[-] (a2) -- (b1);
	
	\end{tikzpicture}
}

\newcommand{\hidden}[1]{
	\begin{tikzpicture}[scale=.1, baseline=-.25em]	
	%\draw[step=1.0,black,thin] (0,0) grid (#1,1);
	\foreach \i in {1,...,#1}{
		\node[circle, draw=white, fill=tumbluelight, inner sep=1pt] at (\i,0){};
	}
	\end{tikzpicture}
}

\newcommand{\drawvector}[1]{
	\begin{tikzpicture}[scale=.1, baseline=-.25em]	
	%\draw[step=1.0,black,thin] (0,0) grid (#1,1);
	\foreach \i in {1,...,#1}{
		\node[circ] at (\i,0){};
	}
	\end{tikzpicture}
}