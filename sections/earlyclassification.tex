
{\setbeamercolor{background canvas}{bg=tumbluedark}
	\begin{frame}[plain]
	
	\vfill
	\Huge\color{white}
	\begin{center}
		\begin{columns}
			\column{.5\textwidth}
			\vspace{7em}
			
			\hfill 
			Early Classification
			\column{.5\textwidth}
			
			%\includegraphics[width=7cm]{images/fdl}
		\end{columns}
	\end{center}
	
	\vfill
\end{frame}
}



\tikzstyle{rnn}=[draw,circle]
\tikzstyle{annot}=[rounded corners, fill=colorblue!20]
\colorlet{colortrain}{colorblue}
\colorlet{colorinfer}{colororange}
\tikzstyle{infer}=[-stealth, shorten >=.0em, shorten <=.0em, colorinfer]
\tikzstyle{loss}=[fill=colorblue!10, rounded corners, font=\small]
\tikzstyle{grad}=[colortrain]

\newcommand{\classimagepair}[1]{
\def\sample{#1}
\begin{tikzpicture}[node distance=.2em]
\node[label=above:{inputs $\V{x}_t$}](a){\includegraphics[width=.5\textwidth]{images/classification_without_earliness/TwoPatterns30EpochsNoEarliness/sample_\sample_x.png}};
\node[label=above:{softmaxed class scores $\yhat_t$}, right=of a](b){\includegraphics[width=.5\textwidth]{images/classification_without_earliness/TwoPatterns30EpochsNoEarliness/sample_\sample_p(y).png}};

\visible<2->{
%\draw (-2,-2) to[grid with coordinates] (8,4);
\node[annot, yshift=3em](wiggle1) at ($ (a)!0.3!(b) $) {event \#1};
\draw (wiggle1) -- (-1.5,0);
\draw (wiggle1) -- (5.5,-1);
}
\visible<3->{
\node[annot, yshift=-5em](wiggle2) at ($ (a)!0.3!(b) $) {event \#2};
\draw (wiggle2) -- (1,0);
\draw (wiggle2) -- (7.5,0);
}

\visible<4->{
\draw[very thick] (8.5,-2) -- (8.5,2); 
\node[annot, yshift=-5em, anchor=east](stop) at (10,-1) {...we could stop here};
%		\draw[shorten >=1em] (stop)++(2,0.5) -- (8,0);
}
\end{tikzpicture}
}

\begin{frame}<presentation:1-4>{Class Predictions}
\classimagepair{0}
\end{frame}


\begin{frame}
\frametitle{Early Classification on Remote Sensing Data}
\documentclass[lang=english]{tumarxivarticle}

\usepackage[
    backend=biber,
    style=numeric,
    maxbibnames=99,
    maxcitenames=1,
    doi=false,
    isbn=false,
    url=false,
    date=year,
    sorting=ydnt
  ]{biblatex}

\usepackage{filecontents}
\begin{filecontents}{bib.bib}
  @Book{Liebel18,
    author = {Lukas Liebel},
    title  = {The TUM-LMF arXiv Documentclass and Me VI},
    year   = {2018},
  }
\end{filecontents}
\bibliography{bib.bib}

\usepackage{blindtext}

\begin{document}

% --------------------------------------------------------------------------------------------------------- HEADER

\title{
  \normalfont
  Documentclass for arXiv Articles
  }

\setkomafont{author}{\large}
\author{
  Lukas Liebel
  }
\publishers{\normalsize
  Computer Vision Research Group, Chair of Remote Sensing Technology \\ Technical University of Munich, Germany \\ lukas.liebel@tum.de \\[0.25cm]
}
\date{}


\twocolumn[
  \maketitle
  \renewcommand{\abstractname}{}
  \begin{onecolabstract}

    \blindtext

  \end{onecolabstract}
  \vspace{1cm}
]

% reset all abbreviations and emphasize all gls entries at first use from now on
\glsresetall
\defglsentryfmt{\ifglsused{\glslabel}{\glsgenentryfmt}{\emph{\glsgenentryfmt}}}


% --------------------------------------------------------------------------------------------------------- CONTENT

\section{Content}
\blindtext

\subsection{Related Work}
\Textcite{Liebel18}

\subsection{More Blindtext}
\blindtext[3]


% --------------------------------------------------------------------------------------------------------- REFERENCES

\printbibliography


\end{document}


\url{https://arxiv.org/abs/1901.10681}

\end{frame}

%\begin{frame}
%	\frametitle{Autoregressive Classification Model}
%	
%	\input{images/classmodel.tikz}
%	
%\end{frame}


{\setbeamercolor{background canvas}{bg=tumbluedark}
	\begin{frame}[plain]
	
	\vfill
	\Huge\color{white}
	\begin{center}
		\begin{columns}
			\column{.5\textwidth}
			\vspace{7em}
			
			\hfill 
			Method
			\column{.5\textwidth}
			
			%\includegraphics[width=7cm]{images/fdl}
		\end{columns}
	\end{center}
	
	\vfill
\end{frame}
}

\begin{frame}
\frametitle{Augmenting Classification Models}

\begin{columns}
	
	\column{.5\textwidth}
	\begin{center}
		
		\begin{tikzpicture}[node distance=1em and 1.5em]
\node[](x0){$x_t$};
\node[rnn, below=of x0](h0){\small$f\left(\xuptot\right)$};
\node[below right= 2em and .0em of h0](y0){$\yhat_t$};
\node[rnn, left=2em of h0,draw=lightgray](hprev){};
\node[rnn, right=2em of h0,draw=lightgray](hnext){};

\draw[infer] (x0) -- (h0);
\draw[infer,draw=lightgray] (hprev) -- (h0);
\draw[infer,draw=lightgray] (h0) -- (hnext);
\draw[infer] (h0) -- (y0) node[midway,right, text=black](wc){$\theta_{cl}$};

\visible<2->{
\node[below left= 2em and .0em of h0](d0){$\delta_t$};
\draw[infer] (h0) -- (d0) node[midway,left, text=black](wd){$\theta_{\delta}$};
}

\visible<3->{
	\node[loss, below=of y0](L0){$\mathcal{L}_{cl}(\yhat,\V{y})$};
	%\node[right=of L0](t0){$\V{y}_t$};
	\draw[-stealth, grad] (y0) -- (L0);
	%\draw[-stealth, grad] (t0) -- (L0);
	%\draw[-stealth, grad] (L0) to [in=-25, out=25, looseness=2] node[midway, right, text=colortrain]{$\frac{\partial\mathcal{L}_{cl}}{\partial\theta_\text{rnn}}$}
	%(h0);
	
	\draw[-stealth, grad] (L0) to [bend right=30] node[near end, right, text=colortrain]{$\frac{\partial\mathcal{L}_{cl}}{\partial\theta_\text{cl}}$}
	(wc);
}

\only<4>{
\node[below=7em of h0, loss](L){$\mathcal{L}(\V{y_t},\yhat_t) = \delta_t\mathcal{L}_{cl}(\V{y_t},\yhat_t)$};
\draw[infer] (L0) -- (L);
\draw[infer] (d0) -- (L);

\draw[-stealth, grad] (L) to [bend right=30] node[midway, right, text=colortrain]{$\frac{\partial\mathcal{L}}{\partial\mathcal{L}_{cl}}$}
(L0);

\draw[-stealth, grad] (L) to [bend left=30] node[midway, left, text=colortrain]{$\frac{\partial\mathcal{L}}{\partial\theta_\delta}$}
(wd);

}

\only<5->{

\node[below=of d0](pt){$P(t)$};
\node[below=8em of h0, loss](L){$\mathcal{L}(\V{y_t},\yhat_t) = P(t)\mathcal{L}_{cl}(\V{y_t},\yhat_t)$};


\draw[infer] (L0) -- (L);
\draw[infer] (d0) -- (pt);
\draw[infer] (pt) -- (L);

%}
%
\only<5->{
\node[left=.5em of pt](budget){\small $\prod_{t'<t}(1-\delta_{t'})$};
\draw[infer] (budget) -- (pt);}

%\only<7->{
%	\node[left=.5em of pt](budget){$\mathcal{B}_{t-1}$};
%	
%}

\draw[-stealth, grad] (L) to [bend right=30] node[midway, right, text=colortrain]{$\frac{\partial\mathcal{L}}{\partial\mathcal{L}_{cl}}$}
(L0);

\draw[-stealth, grad] (L) to [bend left=30] node[midway, left, text=colortrain]{$\frac{\partial\mathcal{L}}{\partial P(t)}$}
(pt);

\draw[-stealth, grad] (pt) to [bend left=30] node[midway, left, text=colortrain]{$\frac{\partial P(t)}{\partial \theta_{\delta}}$}
(wd);

}

\end{tikzpicture}

	\end{center}
	\column{.5\textwidth}
	\begin{tikzpicture}

\pgfplotstableread[col sep = comma]{images/qualitative_example/early_rnn_run-sample0.csv}\mydata

\begin{groupplot}[
	group style={
		group name=my plots,
		group size=1 by 5,
		columns=1,
		xlabels at=edge bottom,
		xticklabels at=edge bottom,
		vertical sep=5pt,
	},
	ylabel near ticks,
	ylabel style={rotate=-90},
	width=\textwidth,
	height=3cm,
	axis x line=bottom,
	axis y line=left,
	enlarge x limits=0.01,
	ymajorgrids,
]

\nextgroupplot[no marks, enlarge y limits=0.05, hide x axis, ylabel={$x$}]
\addplot[thick, tumblue] table[x=t, y=x]{\mydata};

\nextgroupplot[no marks, ylabel={$\yhat_t$}, enlarge y limits=0.05, hide x axis]	
\addplot[thick,colorclassone, name path=y1] table[x=t, y=y1]{\mydata};
\addplot[thick,colorclasstwo, name path=y2] table[x=t, y=y2]{\mydata};
\addplot[thick,colorclassthree, name path=y3] table[x=t, y=y3]{\mydata};
\addplot[thick,colorclassfour, name path=y4] table[x=t, y=y4]{\mydata};

%\addplot[colorblue!20] fill between[of = y1 and axis];
%\addplot[colorhgray!20] fill between[of = y2 and axis];
%\addplot[colorgreen!20] fill between[of = y3 and axis];
%\addplot[colororange!20] fill between[of = y4 and axis];
\only<2->{
\nextgroupplot[ybar, bar width=1pt, ylabel={$\delta_t$}]
\addplot[draw=none, fill=tumblue] table[x=t, y=dt]{\mydata};
}

\only<5->{
\nextgroupplot[ybar, bar width=1pt, hide x axis, ylabel={$P(t)$}]
\addplot[draw=none, fill=tumblue] table[x=t, y=pts]{\mydata};

%\nextgroupplot[ybar, bar width=1pt, hide x axis, ylabel={$\prod_{t'<t}(1-\delta_{t'})$}]
%\addplot[draw=none, fill=tumblue, xlabel={time $t$}] table[x=t, y=Bt]{\mydata};
}

\end{groupplot}

\end{tikzpicture}

	
	
\end{columns}

\end{frame}
%
%\begin{frame}
%\frametitle{Impact of Early Classification on Vegetation Data}
%
%\Large
%
%\begin{itemize}[itemsep=1em]
%\item<1-> \textbf{supervised end-to-end} learning scenario
%\item<2-> we get a stopping time \textbf{for free} solely from classifying labels
%\item<3-> relate to \textbf{characteristic features}, i.e., \textbf{crop phenology}
%\item<4-> next: assess seasonal shifts in \textbf{vegetation phenology} due to \textbf{environmental conditions}
%\end{itemize}
%
%\end{frame}

\begin{frame}
	\frametitle{Loss functions}
	
	{
	\Large
	
	\begin{equation*}
		\mathcal{L}(\M{X}, \V{y}) = \sum_{t=1}^{T} P(t) \left( \mathcal{L}_c (\xuptot, \V{y}) - \mathcal{R}_e(t, \ycorrect_t) \right)
	\end{equation*} 
	
	earlyness reward: $ \mathcal{R}_e(t, \ycorrect_t) = \ycorrect_t \left(1 - \frac{t}{T}\right) $	
	classificaiton loss: $-\log(\ycorrect_t)$ (aka. cross entropy)
	
	}

	\vspace{1em}

	\Large
	\begin{itemize}
		\item $\M{X}=(\V{x}_1,\V{x}_2,\dots,\V{x}_T)$: entire time series of observations $\V{x}$
		\item $\xuptot=(\V{x}_1,\V{x}_2,\dots,\V{x}_t)$: time series until time $t$
		\item $\V{y} \in \{0,1\}^C$: one-hot vector of the classes
		\item $\ycorrect \in [0,1]$: prediction score of the correct class
	\end{itemize}
	
	
\end{frame}

{\setbeamercolor{background canvas}{bg=tumbluedark}
	\begin{frame}[plain]
	
	\vfill
	\Huge\color{white}
	\begin{center}
		\begin{columns}
			\column{.5\textwidth}
			\vspace{7em}
			
			\hfill 
			Results
			\column{.5\textwidth}
			
			%\includegraphics[width=7cm]{images/fdl}
		\end{columns}
	\end{center}
	
	\vfill
\end{frame}
}

%\begin{frame}
%
%\includegraphics[width=\textwidth]{mp4/run_twophase_fast.png}
%\end{frame}

\begin{frame}
	
%	\includegraphics[width=\textwidth]{mp4/run_earyreward_fast.png}
	
	\includemedia[
	width=\textwidth,
	activate=pageopen,
	addresource=mp4/run_earyreward_fast.mp4,
	flashvars={source=mp4/run_earyreward_fast.mp4&loop=true&
		autoPlay=true}
	]{\includegraphics[width=\textwidth]{mp4/run_earyreward_fast.png}}{mp4/run_earyreward_fast.mp4}
	
	
\end{frame}



\begin{frame}
	
	
	\newcommand{\figearlyreward}{
\def\datapath{images/logs/data/early_reward_p2/classes}
\tikzsetnextfilename{loss-accuracyplots}
\begin{tikzpicture}
	\begin{groupplot}[
	group style = {
		group size = 1 by 2,
		x descriptions at=edge bottom,
		vertical sep=1em},
	no markers,
	height=4cm,
%	legend style={at={(0.03,0.5)},anchor=west},
	width=\linewidth,
	xlabel=epochs
 	]
	\nextgroupplot[legend columns=3, ylabel=loss]

	\addplot[test, tumblack] table [x=epoch, y=loss, col sep=comma] {\datapath/log_earliness_test.csv};
	
	\addplot[test, accuracycolor] table [x=epoch, y=loss_classification, col sep=comma] {\datapath/log_earliness_test.csv};
	
	\addplot[test, earlinesscolor] table [x=epoch, y=earliness_reward, col sep=comma] {\datapath/log_earliness_test.csv};
	
%	\legend{total loss (eval), classification loss (eval), earliness reward (eval)}
%	
	\legend{$(\mathcal{L}_\text{c} - \mathcal{R}_\text{e})$, $\mathcal{L}_\text{c}$, $\mathcal{R}_\text{e}$}
	
	
	
	\nextgroupplot[
		legend columns=2, 
		legend pos=south east,
		ytick={0,0.5,1},
		yticklabels={
			0 ({\color{earlinesscolor} jan}),
			0.5 ({\color{earlinesscolor} jun}),
			1 ({\color{earlinesscolor} dec})},
		ylabel=accuracy/$\meantstop$]
	\addplot[test, accuracycolor] table [x=epoch, y=accuracy, col sep=comma] {\datapath/log_earliness_test.csv};
	\addplot[test, earlinesscolor] table [x=epoch, y=earliness, col sep=comma] {\datapath/log_earliness_test.csv};

	\legend{accuracy,$\meantstop$}
	
	\end{groupplot}
\end{tikzpicture}

}

\newcommand{\figtwophasecrossentropy}{
\begin{tikzpicture}
\begin{groupplot}[
group style = {
	group size = 1 by 2,
	x descriptions at=edge bottom,
	vertical sep=1em},
no markers,
height=2.8cm,
%	legend style={at={(0.03,0.5)},anchor=west},
width=\linewidth,
ymin=0,ymax=2.5,
xlabel=epochs
]
\nextgroupplot[legend columns=3, ylabel=loss,
legend style={at={(1,.85)}, anchor=north east}]

\addplot[test, tumblack] table [x=epoch, y=loss, col sep=comma] {images/logs/data/twophasecrossentropy/log_earliness_test.csv};

\addplot[test, accuracycolor] table [x=epoch, y=loss_classification, col sep=comma] {images/logs/data/twophasecrossentropy/log_earliness_test.csv};

\addplot[test, earlinesscolor] table [x=epoch, y=loss_earliness, col sep=comma] {images/logs/data/twophasecrossentropy/log_earliness_test.csv};

\draw[draw, thick] (axis cs:50,0) -- (axis cs:50,2) node[above, font=\sffamily\scriptsize](s){switch epoch};
\node[left=1em of s, font=\sffamily\scriptsize]{accuracy only};
\node[right=1em of s, font=\sffamily\scriptsize]{accuracy+earliness};

\legend{$(\mathcal{L}_\text{c} + \mathcal{L}_\text{e})$, $\mathcal{L}_\text{c}$, $\mathcal{L}_\text{e}$}

\nextgroupplot[
legend columns=2, 
ytick={0,0.5,1},
ymin=0,ymax=1,
legend style={at={(1,0)}, anchor=south east},
yticklabels={
	0 ({\color{earlinesscolor} jan}),
	0.5 ({\color{earlinesscolor} jun}),
	1 ({\color{earlinesscolor} dec})},
ylabel=accuracy/$\meantstop$]
\addplot[test, accuracycolor] table [x=epoch, y=accuracy, col sep=comma] {images/logs/data/twophasecrossentropy/log_earliness_test.csv};
\addplot[test, earlinesscolor] table [x=epoch, y=earliness, col sep=comma] {images/logs/data/twophasecrossentropy/log_earliness_test.csv};

\legend{accuracy,$\meantstop$}

\draw[draw, thick] (axis cs:50,0) -- (axis cs:50,1.1);

\end{groupplot}
\end{tikzpicture}

}


	\begin{columns}
%	\column{.5\textwidth}
%\figtwophasecrossentropy
%{two-phase training first optimizing for accuracy-only and then for early classifications as well using \emph{lateness penalty} loss.}

	\column{\textwidth}
\figearlyreward
{one training phase using the \emph{earliness reward} formulation.}
\end{columns}
	
\end{frame}

\begin{frame}
	
	
		\def\datapath{images/logs/data/early_reward_p2/classes}

\tikzsetnextfilename{trainingstoppingclasses}
\begin{tikzpicture}
	\begin{groupplot}[
	group style = {
		group size = 1 by 7,
		x descriptions at=edge bottom,
		vertical sep=.75em},
	title style={
		font=\sffamily\scriptsize,
		at={(0,1)},
		anchor=north west
	},
	ymajorgrids,
	no markers,
	height=2.7cm,
	legend style={at={(0.5,1)},anchor=south},
	width=\linewidth,
	xlabel=epochs,
	ymax=1.1,
	ymin=0.3,
	ylabel=$\tstop$,
	ytick={0,0.25,0.5,0.75,1},
	yticklabels={Jan,Apr,Jun,Sep,Dec}
	]
	\nextgroupplot[title=meadows, legend columns=3]
	\addplot[meancolor] table [x=epoch, y=meadows, col sep=comma] {\datapath/mean.csv};
	\addplot[mediancolor] table [x=epoch, y=meadows, col sep=comma] {\datapath/median.csv};
	\addplot+[name path=upper, draw=none,forget plot] table [x=epoch, y=meadows, col sep=comma] {\datapath/mean+std.csv};
	\addplot+[name path=lower, draw=none,forget plot] table [x=epoch, y=meadows, col sep=comma] {\datapath/mean-std.csv};
	\addplot[stdcolor] fill between[of=lower and upper];
	
	\legend{mean, median, mean $\mp$ std}
	
	\nextgroupplot[title=winter barley]
	\addplot[meancolor] table [x=epoch, y=winter barley, col sep=comma] {\datapath/mean.csv};
	\addplot[mediancolor] table [x=epoch, y=winter barley, col sep=comma] {\datapath/median.csv};
	\addplot+[name path=upper, draw=none] table [x=epoch, y=winter barley, col sep=comma] {\datapath/mean+std.csv};
	\addplot+[name path=lower, draw=none] table [x=epoch, y=winter barley, col sep=comma] {\datapath/mean-std.csv};
	\addplot[stdcolor] fill between[of=lower and upper];
	\nextgroupplot[title=corn]
%	\addplot table [x=epoch, y=corn, col sep=comma] {images/logs/data/early_reward/classes/mean.csv};
	
	\addplot[meancolor] table [x=epoch, y=corn, col sep=comma] {\datapath/mean.csv};
	\addplot[mediancolor] table [x=epoch, y=corn, col sep=comma] {\datapath/median.csv};
	\addplot+[name path=upper, draw=none] table [x=epoch, y=corn, col sep=comma] {\datapath/mean+std.csv};
	\addplot+[name path=lower, draw=none] table [x=epoch, y=corn, col sep=comma] {\datapath/mean-std.csv};
	\addplot[stdcolor] fill between[of=lower and upper];
	
	\nextgroupplot[title=winter wheat]
%	\addplot table [x=epoch, y=winter wheat, col sep=comma] {images/logs/data/early_reward/classes/mean.csv};
%	
	\addplot[meancolor] table [x=epoch, y=winter wheat, col sep=comma] {\datapath/mean.csv};
	\addplot[mediancolor] table [x=epoch, y=winter wheat, col sep=comma] {\datapath/median.csv};
	\addplot+[name path=upper, draw=none] table [x=epoch, y=winter wheat, col sep=comma] {\datapath/mean+std.csv};
	\addplot+[name path=lower, draw=none] table [x=epoch, y=winter wheat, col sep=comma] {\datapath/mean-std.csv};
	\addplot[stdcolor] fill between[of=lower and upper];
	
	\nextgroupplot[title=summer barley]
%	\addplot table [x=epoch, y=winter barley, col sep=comma] {images/logs/data/early_reward/classes/mean.csv};
	
	\addplot[meancolor] table [x=epoch, y=summer barley, col sep=comma] {\datapath/mean.csv};
	\addplot[mediancolor] table [x=epoch, y=summer barley, col sep=comma] {\datapath/median.csv};
	\addplot+[name path=upper, draw=none] table [x=epoch, y=summer barley, col sep=comma] {\datapath/mean+std.csv};
	\addplot+[name path=lower, draw=none] table [x=epoch, y=summer barley, col sep=comma] {\datapath/mean-std.csv};
	\addplot[stdcolor] fill between[of=lower and upper];
	
	\nextgroupplot[title=clover]
%	\addplot table [x=epoch, y=clover, col sep=comma] {images/logs/data/early_reward/classes/mean.csv};
	
	\addplot[meancolor] table [x=epoch, y=clover, col sep=comma] {\datapath/mean.csv};
	\addplot[mediancolor] table [x=epoch, y=clover, col sep=comma] {\datapath/median.csv};
	\addplot+[name path=upper, draw=none] table [x=epoch, y=clover, col sep=comma] {\datapath/mean+std.csv};
	\addplot+[name path=lower, draw=none] table [x=epoch, y=clover, col sep=comma] {\datapath/mean-std.csv};
	\addplot[stdcolor] fill between[of=lower and upper];
	
	\nextgroupplot[title=winter triticale]
%	\addplot table [x=epoch, y=winter triticale, col sep=comma] {images/logs/data/early_reward/classes/mean.csv};

	\addplot[meancolor] table [x=epoch, y=winter triticale, col sep=comma] {\datapath/mean.csv};
	\addplot[mediancolor] table [x=epoch, y=winter triticale, col sep=comma] {\datapath/median.csv};
	\addplot+[name path=upper, draw=none,forget plot] table [x=epoch, y=winter triticale, col sep=comma] {\datapath/mean+std.csv};
	\addplot+[name path=lower, draw=none,forget plot] table [x=epoch, y=winter triticale, col sep=comma] {\datapath/mean-std.csv};
	\addplot[stdcolor] fill between[of=lower and upper];

	
	\end{groupplot}
	\end{tikzpicture}
		{Class-wise analysis of the estimated stopping time during training in the evaluation set.}
	
\end{frame}

\begin{frame}
\frametitle{Stopping times per crop Class}

%\tikzsetnextfilename{classboxplots}

\tikzstyle{druschdatum} = [thin, star,star points=3, star point ratio=0.5, inner sep=.15em, draw=tumwhite, fill=tumblue]

\newcommand{\druschdatum}{
	\begin{tikzpicture}[scale=2, baseline=-.25em, inner sep=0]
	\node[druschdatum, inner sep=.25em]{};
	\end{tikzpicture}
}


\begin{tikzpicture}

\tikzstyle{boxstyle}=[
	mark options={
	draw=tumblue,
	scale=0.5},
	mark=*,
	solid,
	draw=black]

\begin{axis}[
ytick={1,2,3,4,5,6,7},
yticklabels={
	meadows,
	winter barley,
	corn,
	winter wheat,
	summer barley,
	clover,
	winter triticale},
xmajorgrids,
height=6cm,
xmin = 0,
xmax = 1,
ymin = 1,
ymax = 8,
width=\linewidth,
y axis line style={draw=none},
xtick={0,0.1666666667,0.4166666667,0.6666666667,0.9166666667},
xticklabels={January,March,June,September,December},
xlabel={stopping date $\tstop$}
]


%% august <- recommended end of period
% 0.6666666667
\draw [fill=tumbluelight, opacity=.4, draw=none] (axis cs:0.1666666667,0) rectangle (axis cs:0.5833333333,8);
\draw[draw=tumgraydark] (axis cs:0.5833333333,0) -- (axis cs:0.5833333333,8);

\node[font=\tiny\sffamily] at (axis cs:0.37,8.2){vegetative season};

%\node[font=\tiny\sffamily] at (axis cs:0.8,8.2){b};


\addplot+[boxplot, fill=meadowcolor, boxstyle] table[x = meadows, col sep=comma]{images/logs/data/early_reward_p2/classes/meadows.csv};
%
%\addplot+[fill=meadowcolor, draw=black,
%boxplot prepared={
%median=0.4857142857142857,
%upper quartile=0.4,
%lower quartile=0.5857142857142857,
%upper whisker=0.8642857142857143,
%lower whisker=0.12142857142857144,
%} ] coordinates {};

\addplot+[boxplot, fill=wbarleycolor, boxstyle] table[x=winter barley, col sep=comma]{images/logs/data/early_reward_p2/classes/winter_barley.csv};


%
%\addplot+[fill=wbarleycolor, draw=black,
%boxplot prepared={
%median=0.42857142857142855,
%upper quartile=0.37142857142857144,
%lower quartile=0.5857142857142857,
%upper whisker=0.9071428571428573,
%lower whisker=0.04999999999999999,
%} ] coordinates {};


\addplot+[boxplot, fill=corncolor, boxstyle] table[x =corn, col sep=comma]{images/logs/data/early_reward_p2/classes/corn.csv};
%
%\addplot+[fill=corncolor, draw=black,
%boxplot prepared={
%median=0.45714285714285713,
%upper quartile=0.42857142857142855,
%lower quartile=0.5142857142857142,
%upper whisker=0.6428571428571428,
%lower whisker=0.30000000000000004,
%} ] coordinates {};

\addplot+[boxplot, fill=wheatcolor, boxstyle] table[x =winter wheat, col sep=comma]{images/logs/data/early_reward_p2/classes/winter_wheat.csv};


%
%\addplot+[fill=wheatcolor, draw=black,
%boxplot prepared={
%median=0.5285714285714286,
%upper quartile=0.45714285714285713,
%lower quartile=0.5857142857142857,
%upper whisker=0.7785714285714287,
%lower whisker=0.26428571428571423,
%} ] coordinates {};

\addplot+[boxplot, fill=sbarleycolor, boxstyle] table[x =summer barley, col sep=comma]{images/logs/data/early_reward_p2/classes/summer_barley.csv};

%\addplot+[fill=sbarleycolor, draw=black,
%boxplot prepared={
%median=0.37142857142857144,
%upper quartile=0.2857142857142857,
%lower quartile=0.5428571428571428,
%upper whisker=0.9285714285714285,
%lower whisker=-0.09999999999999998,
%} ] coordinates {};

\addplot+[boxplot, fill=clovercolor, boxstyle] table[x=clover, col sep=comma]{images/logs/data/early_reward_p2/classes/clover.csv};

%\addplot+[fill=clovercolor, draw=black,solid,
%boxplot prepared={
%median=0.5,
%upper quartile=0.42857142857142855,
%lower quartile=0.6142857142857143,
%upper whisker=0.892857142857143,
%lower whisker=0.14999999999999986,
%} ] coordinates {};

\addplot+[boxplot, fill=triticalecolor, boxstyle] table[x=winter triticale, col sep=comma]{images/logs/data/early_reward_p2/classes/winter_triticale.csv};

\def\triticale{0.5722222222}
\def\sbarley{0.5833333333}
\def\wbarley{0.5388888889}
\def\wheat{0.5694444444}


% triticale drusch datum 26.07.
%\draw (axis cs:0.524537037,6.5) -- (axis cs:0.524537037,7.5);
\node[druschdatum] at (axis cs:\triticale,7){};

%  summer barley drusch datum 1.8.
%\draw (axis cs:\sbarley,4.5) -- (axis cs:\sbarley,5.5);
\node[druschdatum] at (axis cs:\sbarley,5){};


% winter wheat drusch datum 26.07.
%\draw (axis cs:\wheat,3.5) -- (axis cs:\wheat,4.5);
\node[druschdatum] at (axis cs:\wheat,4){};


% winter barley datum 15.07.
%\draw (axis cs:\wbarley,1.5) -- (axis cs:\wbarley,2.5);
\node[druschdatum] at (axis cs:\wbarley,2){};


%\addplot+[fill=triticalecolor, draw=black,solid,
%boxplot prepared={
%median=0.5571428571428572,
%upper quartile=0.5142857142857142,
%lower quartile=0.6142857142857143,
%upper whisker=0.7642857142857145,
%lower whisker=0.3642857142857141,
%} ] coordinates {};

\end{axis}
\end{tikzpicture} 

\end{frame}


\begin{frame}
	\frametitle{ICML AI for Social Good Workshop}
	\begin{columns}
	\column{.5\textwidth}
	\includegraphics[width=.6\textwidth]{images/AI4SG_Poster}
	
	\column{.5\textwidth}
	\includegraphics[width=.6\textwidth]{images/AI4SG_Paper}
	
	\end{columns}
\end{frame}


{\setbeamercolor{background canvas}{bg=tumbluedark}
	\begin{frame}[plain]
	
	\vfill
	\Huge\color{white}
	\begin{center}
		\begin{columns}
			\column{.5\textwidth}
			\vspace{7em}
			
			\hfill 
			Crop Type Data
			\column{.5\textwidth}
			
			%\includegraphics[width=7cm]{images/fdl}
		\end{columns}
	\end{center}
	
	\vfill
\end{frame}
}


\begin{frame}
\frametitle{Building Large-Scale Crop Type Mapping Datasets}

\begin{columns}

\column{.5\textwidth}

\Large

\begin{description}\setlength\itemsep{1em}
	\item[\color{tumblue}collected] yearly within European \textbf{Common Agricultural Policy} (CAP)
	\item[\color{tumblue}declared] by Farmers at \textbf{crop subsidy} applications
	\item[\color{tumblue}today] slowly made publicly available (on a national basis by French \includegraphics[height=.9em]{images/IGN-logo}, Bavarian Stmelf \includegraphics[height=.9em]{images/stmelf-logo}, etc.)
	\item[\color{tumblue}in future] further harmonized within \textbf{Europe's INSPIRE} directive
\end{description}

\column{.5\textwidth}
\includegraphics[width=\textwidth]{images/europe_data2}




\end{columns}

\vspace{2em}

\end{frame}


\begin{frame}
\frametitle{Area of Interest for Early Classification}

\begin{columns}
	\column{.4\textwidth}
	\centering\includegraphics[width=\textwidth]{images/holl.pdf}
	
	The area of interest and partitioning in blocks of 4.5 km for training validation and evaluation
	
	\column{.6\textwidth}
	\tikzsetnextfilename{partition_histograms}
\begin{tikzpicture}
  \centering
  \begin{axis}[
        ybar, axis on top,
        title={},
        height=6cm, width=\textwidth,
        bar width=0.2cm,
        ymajorgrids, tick align=inside,
        major grid style={draw=tumgraylight},
        enlarge y limits={value=.1,upper},
        ymin=0, ymax=60,
        axis x line*=bottom,
        axis y line*=left,
        %ymode=log,
        y axis line style={opacity=0},
        tickwidth=0pt,
        enlarge x limits=true,
        legend style={
            at={(1,0.85)},
            anchor=north east,
            draw=none,
            legend columns=-1,
            rounded corners=0,
            /tikz/every even column/.append style={column sep=0.5cm}
        },
        ylabel={Häufigkeit (\%)},
        symbolic x coords={Wiesen,Sommergerste,Silomais,Winterweizen,Wintergerste,Kleegras,Wintertritikale},
       xtick=data,
       tick label style={rotate=0},
       tick label style={rotate=45,anchor=east},
       ylabel near ticks,
       %nodes near coords={
       % \tiny \pgfmathprintnumber[precision=0]{\pgfplotspointmeta}
       %}
    ]
    \addplot [draw=none, fill=traincolor] coordinates {
(Wiesen, 49.506024096385545)
(Sommergerste, 13.560240963855422)
(Silomais, 9.632530120481928)
(Winterweizen, 8.072289156626505)
(Wintergerste, 8.05421686746988)
(Kleegras, 7.530120481927711)
(Wintertritikale, 3.644578313253012)
  };
   \addplot [draw=none,fill=validcolor] coordinates {
(Wiesen, 49.88550866862938)
(Sommergerste, 13.706247955511941)
(Silomais, 9.846254497873733)
(Winterweizen, 8.897612037945699)
(Wintergerste, 8.27608766764802)
(Kleegras, 6.182531894013739)
(Wintertritikale, 3.205757278377494)
  };
   \addplot [draw=none, fill=evalcolor] coordinates {
(Wiesen, 57.32753103801357)
(Sommergerste, 11.852041469345961)
(Silomais, 8.524254447715347)
(Winterweizen, 6.962754383719442)
(Wintergerste, 6.220401894278766)
(Kleegras, 4.876487904774095)
(Wintertritikale, 4.236528862152822)
  };

    \legend{train,valid,eval}
  \end{axis}
  \end{tikzpicture}
	{Class distribution in the dataset with block-wise separation of train, valid and evaluation partitions.}
	
\end{columns}
\end{frame}



{\setbeamercolor{background canvas}{bg=white}
	\begin{frame}[plain]
	\vfill
	\begin{center}
		\Huge\color{tumblue}
		
		Here, we hand-selected 7 classes from a small region of interest...
		%		
	\end{center}
	
	
\end{frame}
}

\begin{frame}
\frametitle{Large Scale Regional Variations}

\includegraphics[width=5cm]{images/Bavaria}
\includegraphics[width=8cm]{images/Large1954_cerial_growth_stages.png}

\raggedleft { \small Large, E. C. (1954). Growth stages in \\ cereals illustration of the Feekes scale. Plant pathology, 3(4), 128-129. }
\end{frame}

\begin{frame}
	\frametitle{Large Scale Regional Variations}
	
	\begin{columns}
		\column{.5\textwidth}
			\includegraphics[width=\textwidth]{images/France}
		\column{.5\textwidth}
		\Large
		
		Questions:
			\begin{itemize}
				\item how do we learn models on these inter-regional scales?
				\item the same class label will correspond to different representations in the data.
			\end{itemize}
	\end{columns}
	
\end{frame}

\begin{frame}
	\frametitle{Outlook}
	
	\Large
	
	Goals:
	\begin{itemize}
		\item large-scale domain adaptation between regions of multi-temporal vegetation data
		\item addressing long-tailed class distributions of $>$300 distinct (overlapping) categories with 90\% of data in $<$20 classes
	\end{itemize}

	Short-Term Objective:
	\begin{itemize}
		\item compile a large-scale inter-regional crop type mapping dataset to be able to evaluate these questions
	\end{itemize}
\end{frame}

\begin{frame}
	\frametitle{The Goal}
	\includegraphics[width=\textwidth]{images/generalization}
\end{frame}


{\setbeamercolor{background canvas}{bg=black}
	\begin{frame}[plain]
	\vfill
	\begin{center}
		\Huge\color{tumwhite}
		\only<1>{
			\textbf{Thank you!}
			\includegraphics[width=6cm]{images/epic1}
		}
%		\only<2>{
%			\textbf{Questions/Input?}
%			\includegraphics[width=7cm]{images/France_white}
%		}
		%		\only<3>{
		%			\textbf{Collaborations?}
		%			\includegraphics[width=7cm]{images/France_white}
		%		}
		%		
	\end{center}
	
	
%	\vspace{1em}	
%	\color{white}
%	{Twitter}: \textbf{marccoru} | {Github}: \textbf{marccoru} or \textbf{tum-lmf} | \includegraphics[height=.9em]{images/TUM-white}~{Chair} \textbf{lmf.bgu.tum.de/vision} \\ 
%	
%	\vspace{1em}
%	
%	\Large
%	\url{https://marccoru.github.io/}
	
	\vfill
\end{frame}
}


{\setbeamercolor{background canvas}{bg=black}
	\begin{frame}[plain]
	\vfill
	\begin{center}
		\Huge\color{tumwhite}
		\only<1>{
			\textbf{backup slides...}
		}
			
	\end{center}
	
	
	\vfill
\end{frame}
}



\begin{frame}
\frametitle{BreizhCrops Dataset (ICML Workshop Submission)}

\begin{center}
	\includegraphics[width=3cm]{images/map/europe}
	\includegraphics[width=3cm]{images/map/regions}
	\includegraphics[width=3cm]{images/map/breizh}
	\includegraphics[width=3cm]{images/map/parcels}
\end{center}

%\vspace{1em}

\documentclass[lang=english]{tumarxivarticle}

\usepackage[
    backend=biber,
    style=numeric,
    maxbibnames=99,
    maxcitenames=1,
    doi=false,
    isbn=false,
    url=false,
    date=year,
    sorting=ydnt
  ]{biblatex}

\usepackage{filecontents}
\begin{filecontents}{bib.bib}
  @Book{Liebel18,
    author = {Lukas Liebel},
    title  = {The TUM-LMF arXiv Documentclass and Me VI},
    year   = {2018},
  }
\end{filecontents}
\bibliography{bib.bib}

\usepackage{blindtext}

\begin{document}

% --------------------------------------------------------------------------------------------------------- HEADER

\title{
  \normalfont
  Documentclass for arXiv Articles
  }

\setkomafont{author}{\large}
\author{
  Lukas Liebel
  }
\publishers{\normalsize
  Computer Vision Research Group, Chair of Remote Sensing Technology \\ Technical University of Munich, Germany \\ lukas.liebel@tum.de \\[0.25cm]
}
\date{}


\twocolumn[
  \maketitle
  \renewcommand{\abstractname}{}
  \begin{onecolabstract}

    \blindtext

  \end{onecolabstract}
  \vspace{1cm}
]

% reset all abbreviations and emphasize all gls entries at first use from now on
\glsresetall
\defglsentryfmt{\ifglsused{\glslabel}{\glsgenentryfmt}{\emph{\glsgenentryfmt}}}


% --------------------------------------------------------------------------------------------------------- CONTENT

\section{Content}
\blindtext

\subsection{Related Work}
\Textcite{Liebel18}

\subsection{More Blindtext}
\blindtext[3]


% --------------------------------------------------------------------------------------------------------- REFERENCES

\printbibliography


\end{document}

\begin{columns}
	\column{.5\textwidth}
	
	\textbf{corn grain and silage}
	\dataexample{images/breizhcrops/example/6139251.csv}
	
	\column{.5\textwidth}
	
	\textbf{temporary meadows}
	\dataexample{images/breizhcrops/example/3685593.csv}
	
\end{columns}

%\vspace{1em}


\begin{columns}
	\column{.5\textwidth}
	
	580k samples of Time Series $\M{X}$ and labels $\V{y}$. \Large \url{https://github.com/TUM-LMF/BreizhCrops}
	
	\column{.5\textwidth}
	
	\small\raggedright
	\textsl{
		Rußwurm M., Lefèvre S., and Körner M (2019). \textbf{BreizhCrops: A Satellite Time Series Dataset for Crop Type Identification}. ICML 2019 Time Series Workshop. arXiv:1905.11893
	}
	
\end{columns}

\end{frame}


\begin{frame}
\frametitle{Two Baseline Models}

\Large
Inspired by Models used in NLP, we implemented a \textbf{multi-layer LSTM} and a \textbf{(minified) Transformer encoder}.

\vspace{1em}
\normalsize

%	\begin{table*}[b]
%		\caption{Accuracy metrics for the Multi-layer bidirectional LSTM \cite{hochreiter1997long} and the Transformer-Encoder \cite{Vaswani:transformer}.}
%		\label{tab:accuracies}
%		\centering
\begin{tabular}{lrrrrrrr}
\toprule
baseline & accuracy & $\kappa$ & mean f1 & mean precision  & mean recall \\
\cmidrule(lr){1-1}\cmidrule(lr){2-2}\cmidrule(lr){3-3}\cmidrule(lr){4-4}\cmidrule(lr){5-5}\cmidrule(lr){6-6}\cmidrule(lr){7-7}
Transformer {\small (Vaswani et al., 2017)} & \textbf{0.69}  &  \textbf{0.63} & 0.57 & {0.60} & 0.56 \\
LSTM {\small (Hochreiter and Schmidhuber, 1997)} & 0.68 & 0.62 & \textbf{0.59} & \textbf{0.63} & \textbf{0.58} \\
\bottomrule
\end{tabular}

\vspace{1em}

\Large
\textbf{Takeaway:} 
\begin{itemize}
\item models perform quite similar
\item potential for improvement
\item well-defined classes accurately classified
\item broadly defined classes systematically confused
\end{itemize}
%	\end{table*}

\end{frame}

\begin{frame}
\frametitle{Challenges and Impact}

\Large

\begin{columns}[t]

\column{.5\textwidth}

\visible<1->{
\Large\textbf{Impact}
\vspace{1em}

\begin{description}[itemsep=.5em]
	\item large scale \bluebf{real-world dataset}
	\item effectively \bluebf{unlimited data} (spatially and temporally)
	\item \bluebf{assessing generalization} over large regions
	\item extraction for further \bluebf{vegetation characteristics} in future work (drought indicator, early classification, crop yield)
\end{description}
}

\column{.5\textwidth}

\visible<2->{
\Large\textbf{Challenges}
\vspace{1em}

\begin{description}[itemsep=.5em]
	\item \bluebf{imbalanced} class \bluebf{labels} (>300 raw classes)
	\item classes with \bluebf{similar characteristics}
	\item non-Gaussian noise induced by \bluebf{clouds}
	\item \bluebf{regional} \bluebf{variations} in the class distributions
	\item \bluebf{spatial} \bluebf{autocorrelation}
	\item \bluebf{irregular} temporal \bluebf{sampling} distance
	\item \bluebf{variable} \bluebf{sequence} length
\end{description}
}


\end{columns}
\end{frame}


\newcommand{\good}[1]{\textbf{\color{tumblue}#1}}
\newcommand{\bad}[1]{\textbf{\color{tumorange}#1}}


\newcommand{\confmatgaf}[4]{

\begin{tikzpicture}

  \def\vmax{#3}
  \def\dataindex{#2}
  \def\nclasses{#4}
  
%
%  \pgfplotsset{every axis label/.append style={font=\footnotesize},tick pos=right, ylabel near ticks}
%  
%  \pgfplotsset{
%    axis line style={%
%      opacity=0 
%    }   
%  }

  \begin{groupplot}[
  	group style={
  		group size=2 by 1,
  		xlabels at=edge bottom,
  		ylabels at=edge left,
  		xticklabels at=edge bottom,
  		vertical sep=35pt,
  		group name=seq_len_plot
  	},
  	axis line style={draw=none},
  	title style={yshift=.75em,},
    width=.6\textwidth,
    height=.5\textwidth,
    enlargelimits=false,
    xtick=data,
%     ymin=1,
    xtick distance=1,
    ytick distance=1,
    colormap={example}{%
		color=(tumwhite)
		color=(tumblue)
	},
    ytick=data,
    ytick align=outside,
    xtick align=outside,
%    tick style={draw=none},
    ytick pos = left,  
    tick label style = {font=\small\sansmath\sffamily},
    %xticklabel = {xshift=-0.75cm}
    yticklabel pos=left,
    %yticklabel near ticks,
    xlabel={Prädizierte Klasse},
%    xlabel style={yshift=-2em},
    x label style={at={(axis description cs:0.5,1.1)},anchor=south},
    y label style={at={(axis description cs:0,.5)},anchor=south},
    ylabel={Wahre Klasse},
%     ylabel near ticks,
%    xticklabels={},
  ]
  \nextgroupplot[
%      yticklabels={
%%      	{sugar beet},
%%      	{summer oat},
%%      	{meadow},
%%      	{rape},
%%      	%{vegetable},
%%      	{hop},
%%      	{winter spelt},
%%      	{winter triticale},
%%      	{beans},
%%      	{peas},
%%      	{potato},
%%      	{soybeans},
%%      	{asparagus},
%%      	{winter wheat},
%%      	{winter barley},
%%      	{winter rye},
%%      	{summer barley},
%%      	{maize}
%      },
      xticklabels={
      	\bad{1},2,\bad{3},\bad{4},5,6,7,\good{8},\good{9},10,11,12,13,14,\bad{15},\good{16},\bad{17},18,19,20,21,22
      },
        yticklabels={
  	1,2,3,4,5,6,7,8,9,10,11,12,13,14,15,16,17,18,19,20,21,22
  	},
     %colorbar style={title={\precisionrecall}, xshift=0cm, font=\footnotesize},
     colorbar right,
     colorbar style={
        	title={}, 
        	font=\footnotesize,
        	%at={(0,1)},
        	anchor=north west,
        	width=8pt,
        	ticklabel pos=right,
        	ticklabel style={xshift=2em},
%        	label style={yshift=-1em},
        	rounded corners=1pt
     },
  ]
  
    \addplot[
      matrix plot,
%      draw=tumwhite,
%       nodes near coords=\coordindex,
%       nodes near coords align={center},
%       nodes near coords style={font=\scriptsize},
        shader=faceted,
        faceted color=tumgraylight!20,
%       shader=faceted interp,
      mesh/cols=\nclasses,
      empty line=scanline,
      point meta=explicit,
      point meta min=0,
      point meta max=\vmax,
    ] table[meta index=\dataindex] {#1};
%   \nextgroupplot[
%       title=2017,
%       %colorbar style={title={\precisionrecall}, xshift=0cm, font=\footnotesize},
%       colorbar right,
%       colorbar style={
%       	title={a}, 
%       	font=\footnotesize,
%       	%at={(0,1)},
%       	anchor=north west,
%       	width=8pt,
%       	ticklabel pos=right,
%       	ticklabel style={xshift=1em},
%       	label style={yshift=1em},
%       	rounded corners=1pt
%       },
%       yticklabels={},
%     xticklabels={
%           	{sug.},
%           	{s. oat},
%           	{mead.},
%           	{rape},
%           	%{vegetable},
%           	{hop},
%           	{w. spelt},
%           	{w. trit.},
%           	{beans},
%           	{peas},
%           	{potato},
%           	{soyb.},
%           	{asp.},
%           	{w. wheat},
%           	{w. barley},
%           	{w. rye},
%           	{s. barley},
%           	{maize}
%     },
%       ]
%    
%    \addplot[
%    matrix plot,
%%    draw=tumwhite,
%    ,   
%    %       nodes near coords=\coordindex,
%    %       nodes near coords align={center},
%    %       nodes near coords style={font=\scriptsize},
%    shader=faceted,
%    faceted color=tumgraylight,
%    %       shader=faceted interp,
%    mesh/cols=17,
%    empty line=scanline,
%    point meta=explicit,
%    point meta min=0,
%    point meta max=\vmax,
%    ] table[meta index=\dataindex] {images/confmat/formatted_grucm2017.csv};
    
  \end{groupplot}
\end{tikzpicture}
}
\begin{frame}
\frametitle{Verwechslungen}

\centering

\vspace{-2em}

\confmatgaf{images/data/TUM_ALL_rnn/npy/confmat_flat.csv}{3}{1}{22}
\end{frame}




\begin{frame}
\frametitle{83 Finale Kategorien}

\includegraphics[width=.49\textwidth]{images/83_confmat1}
\includegraphics[width=.49\textwidth]{images/83_confmat2}

\end{frame}

